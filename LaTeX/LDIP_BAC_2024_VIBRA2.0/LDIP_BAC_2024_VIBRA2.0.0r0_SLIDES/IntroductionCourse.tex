\section{Introduzione}
\begin{frame}[fragile]
\frametitle{Introduzione}
ViBra 2.0 nasce come studio alternativo ad un attuale prodotto, commercializzato da un nostro partner, attualmente sul mercato.\\
Il prodotto attuale (ViBra 1.0) produce variazioni di pressione, applicate con specifiche cupole, sui tessuti del paziente.\\ Tramite le conseguenti vibrazioni stimola i tessuti sottostanti tonificando e migliorandone la circolazione.\\
Il dispositivo possiede 14 uscite sincrone, attivate da una soffiante a canali laterali, che convoglia il flusso in un frazionatore che alterna i flussi entata-uscita della soffiante ad ogni uscita.
\end{frame}

\section{Nuovo design: Intenti e Limitazioni}
\begin{frame}[fragile]
\frametitle{Limitazioni attuali}
Affinché si possa migliorare il prodotto attuale (1.0), elenchiamo alcuni dei limiti attuali:
\newline
\begin{list}{}{}
\item{- Numero di canali: (14) Fisso;}
\item{- Forma d'onda: fissa (pseudo-sinusoidale);}
\item{- Velocità  di pulsazone: Unica per tutti i canali (30-500Hz);}
\item{- Notevole ingombro dovuto alla soffiante e al frazionatore;}
\item{- Notevole costo per soffiante e frazionatore.}
\end{list}
\end{frame}

\begin{frame}[fragile]
\frametitle{Migliorie possibili}
Nel nuovo prodotto si vorrebbe ottenere le seguenti migliorie:
\newline
\begin{list}{}{}
\item{- Numero di canali: (14) Selezionabili separatamente;}
\item{- Forma d'onda: Selezionabile;}
\item{- Visualizzazione della pressione applicata;}
\item{- Velocità  di pulsazone: Separata per tutti i canali (30-500Hz);}
\item{- Riduzione degli ingombri generali rispetto al design 1.0;}
\item{- Riduzione dei costi realizzativi.}
\end{list}
\end{frame}

\section{Progetto di studio}
\begin{frame}[fragile]
\frametitle{Motivazione}
Considerate le limitazioni tecniche del dispositivo attuale e l'impossibilità  di utilizzare trattamenti a pressioni e frequenze diverse per ogni attuatore, si vorrebbe realizzare, tramite attuatore elettroacustico, un sistema a canali indipendenti e parametrizzabili singolarmente.\\ Per questo studio ci limiteremo ad un singolo canale.\\
Il range di pressioni di lavoro dovranno essere compresi tra +/-250 mBar per frequenze di stimolazione comprese tra 30Hz e 500Hz.
\end{frame}

%\subsection{Compiti}
%\begin{frame}[fragile]
%\frametitle{Compiti}
%I seguenti compiti saranno da portare a termine:
%\begin{enumerate}{}{}
%\item Definire le specifiche di sistema in accordo con il relatore;
%\item Definire l'architettura per l'implementazione;
%\item Definire l'hardware necessario all'implementazione;
%\item Valutare l'elettronica necessaria all'attuazione, alla misura e al %controllo delle variabili di funzionamento;
%\item Proporre una configurazione hardware (meccanica ed elettronica) per %realizzare il dispositivo mono/duo-canale;
%\item Realizzare meccanica, elettronica e firmware per un banco dimostrativo %funzionale.
%\end{enumerate}
%\end{frame}

\subsection{Schema di principio}
\begin{frame}[fragile]
\frametitle{Schema di principio}
\begin{figure}
\centering
\subfloat[ALU]{\includegraphics[scale=0.4]{./figures/FigA.png}}\
\end{figure}
\end{frame}

\subsection{Schema di principio}
\begin{frame}[fragile]
\frametitle{Schema di principio}
La realizzazione dovrà indicativamente rappresentare lo schema precedente, che si compone di:
\begin{itemize}{}{}
\item \textbf{Alimentazione (Supply) :} produce tutte le alimentazioni neccessarie a partire da 24VDC (ev. non oltre 48VDC).
\item \textbf{Driver :} idealmente in classe D.
\item \textbf{Display Touch :} interfaccia utente e monitoraggio.
\item \textbf{Sensore di posizione :} misura lo spostamento della membrana.
\item \textbf{Sensori di pressione :} misurano la pressione applicata alle cupole.
\end{itemize}
\end{frame}

\begin{frame}[fragile]
\frametitle{Schema di principio}
\begin{itemize}{}{}
\item \textbf{Controllo :} Controllo closed-loop della pressione di applicazione e limitazione dell'escursione della membrana. Idealmente in forma PID analogico o digitale (su microcontrollore).
\item \textbf{Microcontrollore (uC) :} Gestione interfaccia utente, genera i segnali di stimolo, legge i sensori. Eventualmente integra il controllore.
\item \textbf{Attuatore :} produce la pressione elettroacustica. Idealmente di tipo sub-woofer con alto $x_{max}$.
\end{itemize}
\end{frame}

\subsection{Obiettivi di progetto}
\begin{frame}[fragile]
\frametitle{Obiettivi di progetto}
I seguenti obbiettivi di progetto/tesi dovranno essere raggiunti:
\begin{enumerate}{}{}
\item{Definire le specifiche iniziali;}
\item{Studiare ed implementare tramite stampa 3D la struttura meccanica del blocco di attuazione sulla base dell'attuatore selezionato;}
\item{Studiare, simulare, dimensionare ed implementare una scheda elettronica che racchiuda: Display Touch, microcontrollore, controllore, driver, sensori, alimentazione;}
\item{Realizzare un banco prova/dimostratore;}
\item{Definire delle specifiche finali sotto forma di datasheet;}
\item{Redazione della documentazione relativa al progetto: Poster, Presentazione, Rapporto (\LaTeX).}
\end{enumerate}
\end{frame}

\subsection{Tecnologie}
\begin{frame}[fragile]
\frametitle{Tecnologie}
Tramite questo progetto, i seguenti ambiti e tecnologie saranno affrontati:
\begin{enumerate}{}{}
\item Elettronica analogica e/o mixed signal (Driver/sensorica/PID);
\item Conversione energetica DC/DC;
\item Amplificazione in gamma audio con stadio in classe-D;
\item Programmazione Microcontrollore STM32 (idealmente STM32F469i-DISCO);
\item Disegno CAD e stampa 3D (AutoCad Inventor o FreeCad o altro);
\item Dimensionamento e simulazione elettroacustica con parametri Thiele \& Small (WinISD o simili);
\item Disegno CAD di schematico e PCB (Altium o Eagle o KiCad);
\item Stesura documentazione il \LaTeX (Texmaker o LED o OverLeaf).
\end{enumerate}
\end{frame}
